\documentclass[twoside,letterpaper,12pt]{article}
\usepackage{amsmath}
\setlength{\evensidemargin}{0cm}
\setlength{\oddsidemargin}{0cm}
\setlength{\textwidth}{16cm}
\setlength{\textheight}{22cm}
\setlength{\topmargin}{0cm}

\title{Compton Scattering and Pion Photoproduction\\Event Generator\\\vskip5mm\large\emph{Designed for use with the MAMI A2 Geant4 Simulation}}
\author{Philippe Martel}
\date{18 January 2013}

\begin{document}

\maketitle

\section{Relevant Files}
\begin{itemize}
\item BaseGen.h $\rightarrow$ The base generator class
\item BasePart.h $\rightarrow$ The base particle class
\item CompGen.h $\rightarrow$ The Compton generator class
\item EventGen.cxx $\rightarrow$ The main generator code
\item Makefile
\item out/ $\rightarrow$ The directory for all output files
\item par/Comp* $\rightarrow$ The Compton parameter files
\item par/Pi0P* $\rightarrow$ The neutral pion photoproduction parameter files
\item par/PiPN* $\rightarrow$ The charged pion photoproduction parameter files
\item physics.h $\rightarrow$ Various physical constants
\item Pi0PGen.h $\rightarrow$ The neutral pion photoproduction generator class
\item PiPNGen.h $\rightarrow$ The charged pion photoproduction generator class
\item readme.pdf $\rightarrow$ This file
\item version.txt $\rightarrow$ Details updates to this generator
\end{itemize}

\section{To Compile}
Type 'make' from within this main directory (EventGen).

\section{To Run}
Type './EventGen' from within this main directory, and follow the prompts to select the desired conditions.

\section{Selections}

\subsection{Processes}
\begin{itemize}
\item Comp $\left(\gamma\,p\rightarrow\gamma\,p\right)$
\item Pi0P $\left(\gamma\,p\rightarrow\pi_{0}\,p\right)$
\item PiPN $\left(\gamma\,p\rightarrow\pi_{+}\,n\right)$
\end{itemize}

\subsection{Type}
\begin{itemize}
\item Normal
\item Incoherent
\item Coherent
\end{itemize}

\subsection{Weighting}
\begin{itemize}
\item Isotropic (in CM frame)
\item For Comp - Dispersion (Pasquini)
\item For Pi0P/PiPN - DMT, MAID (MD07), SAID (SN11 and SP09)
\end{itemize}

\subsection{Target}
\begin{itemize}
\item For Normal - proton
\item For Incoherent - 12C (16O in progress)
\item For Coherent - 3He, 4He, 12C, 16O
\end{itemize}

\subsection{Beam Polarization ($P_{B}$)}
\begin{itemize}
\item  Unpolarized $\rightarrow P_{B}^{\,\mathrm{lin}}=0$, $P_{B}^{\,\mathrm{cir}}=0$, $\phi_{B}=0$
\item Linear $\rightarrow 0\leq P_{B}^{\,\mathrm{lin}}\leq 1$, $P_{B}^{\,\mathrm{cir}}=0$, $-180\leq \phi_{B}\leq 180$
\item Circular $\rightarrow P_{B}^{\,\mathrm{lin}}=0$, $-1\leq P_{B}^{\,\mathrm{cir}}\leq 1$, $\phi_{B}=0$
\end{itemize}

\subsection{Target Polarization ($P_{T}$)}
\begin{itemize}
\item Unpolarized $\rightarrow P_{T}^{\,\mathrm{tran}}=0$, $P_{T}^{\,\mathrm{long}}=0$, $\phi_{T}=0$
\item Transverse $\rightarrow -1\leq P_{T}^{\,\mathrm{tran}}\leq 1$, $P_{T}^{\,\mathrm{long}}=0$, $-180\leq \phi_{T}\leq 180$
\item Longitudinal  $\rightarrow P_{T}^{\,\mathrm{tran}}=0$, $-1\leq P_{T}^{\,\mathrm{long}}\leq 1$, $\phi_{T}=0$
\end{itemize}

\subsection{Beam Energy}
\begin{itemize}
\item For Compton - $80-450$ MeV
\item For Pi0P - $145-450$ MeV
\item For PiPN - $155-450$ MeV
\end{itemize}

\subsection{Run Limits}
For isotropic distributions, the only available limit for a `run' is to select the total number of events. For any other distributions, the user has the additional option to limit the `run' by time.

\subsubsection{Run Time}
For this option, the user is required to enter the endpoint energy of the electron beam and the photon energy of the highest counting tagger channel (the last channel that is `on'). This allows for limiting the rate in this channel to 1 MHz.

The user must also enter the average tagging efficiency, the area density of the target (in cm$^{-2}$), and the running time desired (in minutes). For the sake of file sizes, and for importing into Geant4, the generator will stop at one million events and display the running time that correlates to.

\subsubsection{Number of Events}

\section{Parameter Files}
For Normal and Incoherent reactions, parameter files are read in from the 'par' sub-directory, currently in 5 MeV steps. Improvements to the code now allow for any decimal selection, as long as it's within the range of available files. (NOTE: If other energies, or a different Pi0P/PiPN solution, is required please contact me and I can provide the appropriate files)

Description of columns in parameter files (using the observables nomenclature of: L.S. Barker, A. Donnachie, J.K. Storrow, Nucl. Phys. B 95 (1975) 347)
\begin{itemize}
\item Angle $\rightarrow$ lab for Compton, cm for Pi0P/PiPN
\item $\left(\frac{d\sigma}{d\omega}\right)_{\mathrm{unp}}\rightarrow$ unpolarized cross section, also written as DSG, in units of nb for Compton or $\mu$b for Pi0P/PiPN
\item $\Sigma\rightarrow$ beam asymmetry, equal to $-\Sigma_{3}$ for Compton, also written as S
\item T $\rightarrow$ transverse target asymmetry observed in plane transverse to target polarization, equal to zero for Compton
\item P $\rightarrow$ linear beam/transverse target asymmetry observed in plane transverse to target polarization, currently equal to zero for Compton
\item G  $\rightarrow$linear beam/longitudinal target asymmetry, currently equal to zero for Compton
\item H  $\rightarrow$linear beam/transverse target asymmetry observed in plane of target polarization, currently equal to zero for Compton
\item E  $\rightarrow$circular beam/longitudinal target asymmetry, equal to $-\Sigma_{2z}$ for Compton
\item F $\rightarrow$ circular beam/transverse target asymmetry observed in plane of target polarization, equal to $\Sigma_{2x}$ for Compton
\end{itemize}

\begin{flalign}
  \frac{d\sigma}{d\omega}=\left(\frac{d\sigma}{d\omega}\right)_{\mathrm{unp}}\Big\{1&-P_{B}^{\,\mathrm{lin}}\,\Sigma\,\mathrm{cos}\,2\phi+P_{x}\left[P_{B}^{\,\mathrm{cir}}F-P_{B}^{\,\mathrm{lin}}H\,\mathrm{sin}\,2\phi\right] \nonumber \\
  &+P_{y}\left[T-P_{B}^{\,\mathrm{lin}}P\,\mathrm{cos}\,2\phi\right]-P_{z}\left[P_{B}^{\,\mathrm{cir}}E-P_{B}^{\,\mathrm{lin}}G\,\mathrm{sin}\,2\phi\right]\Big\} \nonumber
\end{flalign}
where $\phi=\phi_{B}-\phi_{\pi}$, the azimuthal angle between the plane of linear beam polarization (along the electric field) and the pion. Since:
\begin{flalign}
  P_{x}&=P_{T}^{\,\mathrm{tran}}\,\mathrm{cos}\,\phi \nonumber \\
  P_{y}&=P_{T}^{\,\mathrm{tran}}\,\mathrm{sin}\,\phi \nonumber \\
  P_{z}&=P_{T}^{\,\mathrm{long}} \nonumber
\end{flalign}
where $\phi=\phi_{T}-\phi_{\pi}$, the azimuthal angle between the direction of transverse polarization and the pion.
\begin{flalign}
  \frac{d\sigma}{d\omega}=\left(\frac{d\sigma}{d\omega}\right)_{\mathrm{unp}}\Big\{1&-P_{B}^{\,\mathrm{lin}}\,\Sigma\,\mathrm{cos}\,2(\phi_{B}-\phi) \nonumber \\
  &+P_{T}^{\,\mathrm{tran}}\,\mathrm{cos}\,(\phi_{T}-\phi)\left[P_{B}^{\,\mathrm{cir}}F-P_{B}^{\mathrm{lin}}H\,\mathrm{sin}\,2(\phi_{B}-\phi)\right] \nonumber \\
  &+P_{T}^{\,\mathrm{tran}}\,\mathrm{sin}\,(\phi_{T}-\phi)\left[T-P_{B}^{\,\mathrm{lin}}P\,\mathrm{cos}\,2(\phi_{B}-\phi)\right] \nonumber \\
  &-P_{T}^{\,\mathrm{long}}\left[P_{B}^{\,\mathrm{cir}}E-P_{B}^{\,\mathrm{lin}}G\,\mathrm{sin}\,2(\phi_{B}-\phi)\right]\Big\} \nonumber
\end{flalign}

For Coherent reactions, cross sections are produced using a method provided by R. Miskimen.

\section{Output}
Three root files are produced as output in the 'out' sub-directory. 'hist.root' contains histograms for testing, energy and angular distributions, etc. 'tree.root' is a tree file with the Lorentz vectors of the resulting particles for more detailed testing. 'ntpl.root' is then the ntuple for use in the A2 Geant4 simulation.

\section{For the Future}
Next step will be adding either a GUI interface, or at least the option to read in from a setup file (to eliminate the need to enter the same settings over and over again).

Please feel free to let me know of any errors, concerns, suggestions, or requests. Thanks!

\end{document}
